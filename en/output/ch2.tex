\enableregime[utf]  % use UTF-8

\setupcolors[state=start]
\setupinteraction[state=start, color=middleblue] % needed for hyperlinks

\setuppapersize[letter][letter]  % use letter paper
\setuplayout[width=middle, backspace=1.5in, cutspace=1.5in,
             height=middle, header=0.75in, footer=0.75in] % page layout
\setuppagenumbering[location={footer,center}]  % number pages
\setupbodyfont[11pt]  % 11pt font
\setupwhitespace[medium]  % inter-paragraph spacing

\setuphead[section][style=\tfc]
\setuphead[subsection][style=\tfb]
\setuphead[subsubsection][style=\bf]

% define descr (for definition lists)
\definedescription[descr][
  headstyle=bold,style=normal,align=left,location=hanging,
  width=broad,margin=1cm]

% prevent orphaned list intros
\setupitemize[autointro]

% define defaults for bulleted lists 
\setupitemize[1][symbol=1][indentnext=no]
\setupitemize[2][symbol=2][indentnext=no]
\setupitemize[3][symbol=3][indentnext=no]
\setupitemize[4][symbol=4][indentnext=no]

\setupthinrules[width=15em]  % width of horizontal rules

% for block quotations
\unprotect

\startvariables all
blockquote: blockquote
\stopvariables

\definedelimitedtext
[\v!blockquote][\v!quotation]

\setupdelimitedtext
[\v!blockquote]
[\c!left=,
\c!right=,
before={\blank[medium]},
after={\blank[medium]},
]

\protect

\starttext

\subject{Git Basics}

If you can read only one chapter to get going with Git, this is it.
This chapter covers every basic command you need to do the vast
majority of the things you’ll eventually spend your time doing with
Git. By the end of the chapter, you should be able to configure and
initialize a repository, begin and stop tracking files, and stage
and commit changes. We’ll also show you how to set up Git to ignore
certain files and file patterns, how to undo mistakes quickly and
easily, how to browse the history of your project and view changes
between commits, and how to push and pull from remote repositories.

\subsubject{Getting a Git Repository}

You can get a Git project using two main approaches. The first
takes an existing project or directory and imports it into Git. The
second clones an existing Git repository from another server.

\subsubsubject{Initializing a Repository in an Existing Directory}

If you’re starting to track an existing project in Git, you need to
go to the project’s directory and type

\starttyping
$ git init
\stoptyping

This creates a new subdirectory named .git that contains all of
your necessary repository files — a Git repository skeleton. At
this point, nothing in your project is tracked yet. (See Chapter 9
for more information about exactly what files are contained in the
\type{.git} directory you just created.)

If you want to start version-controlling existing files (as opposed
to an empty directory), you should probably begin tracking those
files and do an initial commit. You can accomplish that with a few
git add commands that specify the files you want to track, followed
by a commit:

\starttyping
$ git add *.c
$ git add README
$ git commit -m 'initial project version'
\stoptyping

We’ll go over what these commands do in just a minute. At this
point, you have a Git repository with tracked files and an initial
commit.

\subsubsubject{Cloning an Existing Repository}

If you want to get a copy of an existing Git repository — for
example, a project you’d like to contribute to — the command you
need is git clone. If you’re familiar with other VCS systems such
as Subversion, you’ll notice that the command is clone and not
checkout. This is an important distinction — Git receives a copy of
nearly all data that the server has. Every version of every file
for the history of the project is pulled down when you run
\type{git clone}. In fact, if your server disk gets corrupted, you
can use any of the clones on any client to set the server back to
the state it was in when it was cloned (you may lose some
server-side hooks and such, but all the versioned data would be
there — see Chapter 4 for more details).

You clone a repository with \type{git clone [url]}. For example, if
you want to clone the Ruby Git library called Grit, you can do so
like this:

\starttyping
$ git clone git://github.com/schacon/grit.git
\stoptyping

That creates a directory named \quotation{grit}, initializes a
\type{.git} directory inside it, pulls down all the data for that
repository, and checks out a working copy of the latest version. If
you go into the new \type{grit} directory, you’ll see the project
files in there, ready to be worked on or used. If you want to clone
the repository into a directory named something other than grit,
you can specify that as the next command-line option:

\starttyping
$ git clone git://github.com/schacon/grit.git mygrit
\stoptyping

That command does the same thing as the previous one, but the
target directory is called mygrit.

Git has a number of different transfer protocols you can use. The
previous example uses the \type{git://} protocol, but you may also
see \type{http(s)://} or \type{user@server:/path.git}, which uses
the SSH transfer protocol. Chapter 4 will introduce all of the
available options the server can set up to access your Git
repository and the pros and cons of each.

\subsubject{Recording Changes to the Repository}

You have a bona fide Git repository and a checkout or working copy
of the files for that project. You need to make some changes and
commit snapshots of those changes into your repository each time
the project reaches a state you want to record.

Remember that each file in your working directory can be in one of
two states: tracked or untracked. Tracked files are files that were
in the last snapshot; they can be unmodified, modified, or staged.
Untracked files are everything else - any files in your working
directory that were not in your last snapshot and are not in your
staging area. When you first clone a repository, all of your files
will be tracked and unmodified because you just checked them out
and haven’t edited anything.

As you edit files, Git sees them as modified, because you’ve
changed them since your last commit. You stage these modified files
and then commit all your staged changes, and the cycle repeats.
This lifecycle is illustrated in Figure 2--1.

\placefigure[here,nonumber]{Figure 2--1. The lifecycle of the status of your files.}{\externalfigure[../figures/18333fig0201-tn.png]}

\subsubsubject{Checking the Status of Your Files}

The main tool you use to determine which files are in which state
is the git status command. If you run this command directly after a
clone, you should see something like this:

\starttyping
$ git status
# On branch master
nothing to commit (working directory clean)
\stoptyping

This means you have a clean working directory — in other words,
there are no tracked and modified files. Git also doesn’t see any
untracked files, or they would be listed here. Finally, the command
tells you which branch you’re on. For now, that is always master,
which is the default; you won’t worry about it here. The next
chapter will go over branches and references in detail.

Let’s say you add a new file to your project, a simple README file.
If the file didn’t exist before, and you run \type{git status}, you
see your untracked file like so:

\starttyping
$ vim README
$ git status
# On branch master
# Untracked files:
#   (use "git add <file>..." to include in what will be committed)
#
#   README
nothing added to commit but untracked files present (use "git add" to track)
\stoptyping

You can see that your new README file is untracked, because it’s
under the “Untracked files” heading in your status output.
Untracked basically means that Git sees a file you didn’t have in
the previous snapshot (commit); Git won’t start including it in
your commit snapshots until you explicitly tell it to do so. It
does this so you don’t accidentally begin including generated
binary files or other files that you did not mean to include. You
do want to start including README, so let’s start tracking the
file.

\subsubsubject{Tracking New Files}

In order to begin tracking a new file, you use the command
\type{git add}. To begin tracking the README file, you can run
this:

\starttyping
$ git add README
\stoptyping

If you run your status command again, you can see that your README
file is now tracked and staged:

\starttyping
$ git status
# On branch master
# Changes to be committed:
#   (use "git reset HEAD <file>..." to unstage)
#
#   new file:   README
#
\stoptyping

You can tell that it’s staged because it’s under the “Changes to be
committed” heading. If you commit at this point, the version of the
file at the time you ran git add is what will be in the historical
snapshot. You may recall that when you ran git init earlier, you
then ran git add (files) — that was to begin tracking files in your
directory. The git add command takes a path name for either a file
or a directory; if it’s a directory, the command adds all the files
in that directory recursively.

\subsubsubject{Staging Modified Files}

Let’s change a file that was already tracked. If you change a
previously tracked file called \type{benchmarks.rb} and then run
your \type{status} command again, you get something that looks like
this:

\starttyping
$ git status
# On branch master
# Changes to be committed:
#   (use "git reset HEAD <file>..." to unstage)
#
#   new file:   README
#
# Changed but not updated:
#   (use "git add <file>..." to update what will be committed)
#
#   modified:   benchmarks.rb
#
\stoptyping

The benchmarks.rb file appears under a section named “Changed but
not updated” — which means that a file that is tracked has been
modified in the working directory but not yet staged. To stage it,
you run the \type{git add} command (it’s a multipurpose command —
you use it to begin tracking new files, to stage files, and to do
other things like marking merge-conflicted files as resolved).
Let’s run \type{git add} now to stage the benchmarks.rb file, and
then run \type{git status} again:

\starttyping
$ git add benchmarks.rb
$ git status
# On branch master
# Changes to be committed:
#   (use "git reset HEAD <file>..." to unstage)
#
#   new file:   README
#   modified:   benchmarks.rb
#
\stoptyping

Both files are staged and will go into your next commit. At this
point, suppose you remember one little change that you want to make
in benchmarks.rb before you commit it. You open it again and make
that change, and you’re ready to commit. However, let’s run
\type{git status} one more time:

\starttyping
$ vim benchmarks.rb
$ git status
# On branch master
# Changes to be committed:
#   (use "git reset HEAD <file>..." to unstage)
#
#   new file:   README
#   modified:   benchmarks.rb
#
# Changed but not updated:
#   (use "git add <file>..." to update what will be committed)
#
#   modified:   benchmarks.rb
#
\stoptyping

What the heck? Now benchmarks.rb is listed as both staged and
unstaged. How is that possible? It turns out that Git stages a file
exactly as it is when you run the git add command. If you commit
now, the version of benchmarks.rb as it was when you last ran the
git add command is how it will go into the commit, not the version
of the file as it looks in your working directory when you run git
commit. If you modify a file after you run \type{git add}, you have
to run \type{git add} again to stage the latest version of the
file:

\starttyping
$ git add benchmarks.rb
$ git status
# On branch master
# Changes to be committed:
#   (use "git reset HEAD <file>..." to unstage)
#
#   new file:   README
#   modified:   benchmarks.rb
#
\stoptyping

\subsubsubject{Ignoring Files}

Often, you’ll have a class of files that you don’t want Git to
automatically add or even show you as being untracked. These are
generally automatically generated files such as log files or files
produced by your build system. In such cases, you can create a file
listing patterns to match them named .gitignore. Here is an example
.gitignore file:

\starttyping
$ cat .gitignore
*.[oa]
*~
\stoptyping

The first line tells Git to ignore any files ending in .o or .a —
object and archive files that may be the product of building your
code. The second line tells Git to ignore all files that end with a
tilde (\type{~}), which is used by many text editors such as Emacs
to mark temporary files. You may also include a log, tmp, or pid
directory; automatically generated documentation; and so on.
Setting up a .gitignore file before you get going is generally a
good idea so you don’t accidentally commit files that you really
don’t want in your Git repository.

The rules for the patterns you can put in the .gitignore file are
as follows:

\startitemize
\item
  Blank lines or lines starting with \# are ignored.
\item
  Standard glob patterns work.
\item
  You can end patterns with a forward slash (\type{/}) to specify a
  directory.
\item
  You can negate a pattern by starting it with an exclamation point
  (\type{!}).
\stopitemize

Glob patterns are like simplified regular expressions that shells
use. An asterisk (\type{*}) matches zero or more characters;
\type{[abc]} matches any character inside the brackets (in this
case a, b, or c); a question mark (\type{?}) matches a single
character; and brackets enclosing characters separated by a
hyphen(\type{[0-9]}) matches any character between them (in this
case 0 through 9) .

Here is another example .gitignore file:

\starttyping
# a comment - this is ignored
*.a       # no .a files
!lib.a    # but do track lib.a, even though you're ignoring .a files above
/TODO     # only ignore the root TODO file, not subdir/TODO
build/    # ignore all files in the build/ directory
doc/*.txt # ignore doc/notes.txt, but not doc/server/arch.txt
\stoptyping

\subsubsubject{Viewing Your Staged and Unstaged Changes}

If the \type{git status} command is too vague for you — you want to
know exactly what you changed, not just which files were changed —
you can use the \type{git diff} command. We’ll cover
\type{git diff} in more detail later; but you’ll probably use it
most often to answer these two questions: What have you changed but
not yet staged? And what have you staged that you are about to
commit? Although \type{git status} answers those questions very
generally, \type{git diff} shows you the exact lines added and
removed — the patch, as it were.

Let’s say you edit and stage the README file again and then edit
the benchmarks.rb file without staging it. If you run your
\type{status} command, you once again see something like this:

\starttyping
$ git status
# On branch master
# Changes to be committed:
#   (use "git reset HEAD <file>..." to unstage)
#
#   new file:   README
#
# Changed but not updated:
#   (use "git add <file>..." to update what will be committed)
#
#   modified:   benchmarks.rb
#
\stoptyping

To see what you’ve changed but not yet staged, type \type{git diff}
with no other arguments:

\starttyping
$ git diff
diff --git a/benchmarks.rb b/benchmarks.rb
index 3cb747f..da65585 100644
--- a/benchmarks.rb
+++ b/benchmarks.rb
@@ -36,6 +36,10 @@ def main
           @commit.parents[0].parents[0].parents[0]
         end

+        run_code(x, 'commits 1') do
+          git.commits.size
+        end
+
         run_code(x, 'commits 2') do
           log = git.commits('master', 15)
           log.size
\stoptyping

That command compares what is in your working directory with what
is in your staging area. The result tells you the changes you’ve
made that you haven’t yet staged.

If you want to see what you’ve staged that will go into your next
commit, you can use \type{git diff --cached}. (In Git versions
1.6.1 and later, you can also use \type{git diff --staged}, which
may be easier to remember.) This command compares your staged
changes to your last commit:

\starttyping
$ git diff --cached
diff --git a/README b/README
new file mode 100644
index 0000000..03902a1
--- /dev/null
+++ b/README2
@@ -0,0 +1,5 @@
+grit
+ by Tom Preston-Werner, Chris Wanstrath
+ http://github.com/mojombo/grit
+
+Grit is a Ruby library for extracting information from a Git repository
\stoptyping

It’s important to note that \type{git diff} by itself doesn’t show
all changes made since your last commit — only changes that are
still unstaged. This can be confusing, because if you’ve staged all
of your changes, \type{git diff} will give you no output.

For another example, if you stage the benchmarks.rb file and then
edit it, you can use \type{git diff} to see the changes in the file
that are staged and the changes that are unstaged:

\starttyping
$ git add benchmarks.rb
$ echo '# test line' >> benchmarks.rb
$ git status
# On branch master
#
# Changes to be committed:
#
#   modified:   benchmarks.rb
#
# Changed but not updated:
#
#   modified:   benchmarks.rb
#
\stoptyping

Now you can use \type{git diff} to see what is still unstaged

\starttyping
$ git diff
diff --git a/benchmarks.rb b/benchmarks.rb
index e445e28..86b2f7c 100644
--- a/benchmarks.rb
+++ b/benchmarks.rb
@@ -127,3 +127,4 @@ end
 main()

 ##pp Grit::GitRuby.cache_client.stats
+# test line
\stoptyping

and \type{git diff --cached} to see what you’ve staged so far:

\starttyping
$ git diff --cached
diff --git a/benchmarks.rb b/benchmarks.rb
index 3cb747f..e445e28 100644
--- a/benchmarks.rb
+++ b/benchmarks.rb
@@ -36,6 +36,10 @@ def main
          @commit.parents[0].parents[0].parents[0]
        end

+        run_code(x, 'commits 1') do
+          git.commits.size
+        end
+
        run_code(x, 'commits 2') do
          log = git.commits('master', 15)
          log.size
\stoptyping

\subsubsubject{Committing Your Changes}

Now that your staging area is set up the way you want it, you can
commit your changes. Remember that anything that is still unstaged
— any files you have created or modified that you haven’t run
\type{git add} on since you edited them — won’t go into this
commit. They will stay as modified files on your disk. In this
case, the last time you ran \type{git status}, you saw that
everything was staged, so you’re ready to commit your changes. The
simplest way to commit is to type \type{git commit}:

\starttyping
$ git commit
\stoptyping

Doing so launches your editor of choice. (This is set by your
shell’s \type{$EDITOR} environment variable — usually vim or emacs,
although you can configure it with whatever you want using the
\type{git config --global core.editor} command as you saw in
Chapter 1).

The editor displays the following text (this example is a Vim
screen):

\starttyping
# Please enter the commit message for your changes. Lines starting
# with '#' will be ignored, and an empty message aborts the commit.
# On branch master
# Changes to be committed:
#   (use "git reset HEAD <file>..." to unstage)
#
#       new file:   README
#       modified:   benchmarks.rb
~
~
~
".git/COMMIT_EDITMSG" 10L, 283C
\stoptyping

You can see that the default commit message contains the latest
output of the \type{git status} command commented out and one empty
line on top. You can remove these comments and type your commit
message, or you can leave them there to help you remember what
you’re committing. (For an even more explicit reminder of what
you’ve modified, you can pass the \type{-v} option to
\type{git commit}. Doing so also puts the diff of your change in
the editor so you can see exactly what you did.) When you exit the
editor, Git creates your commit with that commit message (with the
comments and diff stripped out).

Alternatively, you can type your commit message inline with the
\type{commit} command by specifying it after a -m flag, like this:

\starttyping
$ git commit -m "Story 182: Fix benchmarks for speed"
[master]: created 463dc4f: "Fix benchmarks for speed"
 2 files changed, 3 insertions(+), 0 deletions(-)
 create mode 100644 README
\stoptyping

Now you’ve created your first commit! You can see that the commit
has given you some output about itself: which branch you committed
to (master), what SHA--1 checksum the commit has (\type{463dc4f}),
how many files were changed, and statistics about lines added and
removed in the commit.

Remember that the commit records the snapshot you set up in your
staging area. Anything you didn’t stage is still sitting there
modified; you can do another commit to add it to your history.
Every time you perform a commit, you’re recording a snapshot of
your project that you can revert to or compare to later.

\subsubsubject{Skipping the Staging Area}

Although it can be amazingly useful for crafting commits exactly
how you want them, the staging area is sometimes a bit more complex
than you need in your workflow. If you want to skip the staging
area, Git provides a simple shortcut. Providing the \type{-a}
option to the \type{git commit} command makes Git automatically
stage every file that is already tracked before doing the commit,
letting you skip the \type{git add} part:

\starttyping
$ git status
# On branch master
#
# Changed but not updated:
#
#   modified:   benchmarks.rb
#
$ git commit -a -m 'added new benchmarks'
[master 83e38c7] added new benchmarks
 1 files changed, 5 insertions(+), 0 deletions(-)
\stoptyping

Notice how you don’t have to run \type{git add} on the
benchmarks.rb file in this case before you commit.

\subsubsubject{Removing Files}

To remove a file from Git, you have to remove it from your tracked
files (more accurately, remove it from your staging area) and then
commit. The \type{git rm} command does that and also removes the
file from your working directory so you don’t see it as an
untracked file next time around.

If you simply remove the file from your working directory, it shows
up under the “Changed but not updated” (that is, {\em unstaged})
area of your \type{git status} output:

\starttyping
$ rm grit.gemspec
$ git status
# On branch master
#
# Changed but not updated:
#   (use "git add/rm <file>..." to update what will be committed)
#
#       deleted:    grit.gemspec
#
\stoptyping

Then, if you run \type{git rm}, it stages the file’s removal:

\starttyping
$ git rm grit.gemspec
rm 'grit.gemspec'
$ git status
# On branch master
#
# Changes to be committed:
#   (use "git reset HEAD <file>..." to unstage)
#
#       deleted:    grit.gemspec
#
\stoptyping

The next time you commit, the file will be gone and no longer
tracked. If you modified the file and added it to the index
already, you must force the removal with the \type{-f} option. This
is a safety feature to prevent accidental removal of data that
hasn’t yet been recorded in a snapshot and that can’t be recovered
from Git.

Another useful thing you may want to do is to keep the file in your
working tree but remove it from your staging area. In other words,
you may want to keep the file on your hard drive but not have Git
track it anymore. This is particularly useful if you forgot to add
something to your \type{.gitignore} file and accidentally added it,
like a large log file or a bunch of \type{.a} compiled files. To do
this, use the \type{--cached} option:

\starttyping
$ git rm --cached readme.txt
\stoptyping

You can pass files, directories, and file-glob patterns to the
\type{git rm} command. That means you can do things such as

\starttyping
$ git rm log/\*.log
\stoptyping

Note the backslash (\type{\}) in front of the \type{*}. This is
necessary because Git does its own filename expansion in addition
to your shell’s filename expansion. This command removes all files
that have the \type{.log} extension in the \type{log/} directory.
Or, you can do something like this:

\starttyping
$ git rm \*~
\stoptyping

This command removes all files that end with \type{~}.

\subsubsubject{Moving Files}

Unlike many other VCS systems, Git doesn’t explicitly track file
movement. If you rename a file in Git, no metadata is stored in Git
that tells it you renamed the file. However, Git is pretty smart
about figuring that out after the fact — we’ll deal with detecting
file movement a bit later.

Thus it’s a bit confusing that Git has a \type{mv} command. If you
want to rename a file in Git, you can run something like

\starttyping
$ git mv file_from file_to
\stoptyping

and it works fine. In fact, if you run something like this and look
at the status, you’ll see that Git considers it a renamed file:

\starttyping
$ git mv README.txt README
$ git status
# On branch master
# Your branch is ahead of 'origin/master' by 1 commit.
#
# Changes to be committed:
#   (use "git reset HEAD <file>..." to unstage)
#
#       renamed:    README.txt -> README
#
\stoptyping

However, this is equivalent to running something like this:

\starttyping
$ mv README.txt README
$ git rm README.txt
$ git add README
\stoptyping

Git figures out that it’s a rename implicitly, so it doesn’t matter
if you rename a file that way or with the \type{mv} command. The
only real difference is that \type{mv} is one command instead of
three — it’s a convenience function. More important, you can use
any tool you like to rename a file, and address the add/rm later,
before you commit.

\subsubject{Viewing the Commit History}

After you have created several commits, or if you have cloned a
repository with an existing commit history, you’ll probably want to
look back to see what has happened. The most basic and powerful
tool to do this is the \type{git log} command.

These examples use a very simple project called simplegit that I
often use for demonstrations. To get the project, run

\starttyping
git clone git://github.com/schacon/simplegit-progit.git
\stoptyping

When you run \type{git log} in this project, you should get output
that looks something like this:

\starttyping
$ git log
commit ca82a6dff817ec66f44342007202690a93763949
Author: Scott Chacon <schacon@gee-mail.com>
Date:   Mon Mar 17 21:52:11 2008 -0700

    changed the version number

commit 085bb3bcb608e1e8451d4b2432f8ecbe6306e7e7
Author: Scott Chacon <schacon@gee-mail.com>
Date:   Sat Mar 15 16:40:33 2008 -0700

    removed unnecessary test code

commit a11bef06a3f659402fe7563abf99ad00de2209e6
Author: Scott Chacon <schacon@gee-mail.com>
Date:   Sat Mar 15 10:31:28 2008 -0700

    first commit
\stoptyping

By default, with no arguments, \type{git log} lists the commits
made in that repository in reverse chronological order. That is,
the most recent commits show up first. As you can see, this command
lists each commit with its SHA--1 checksum, the author’s name and
e-mail, the date written, and the commit message.

A huge number and variety of options to the \type{git log} command
are available to show you exactly what you’re looking for. Here,
we’ll show you some of the most-used options.

One of the more helpful options is \type{-p}, which shows the diff
introduced in each commit. You can also use \type{-2}, which limits
the output to only the last two entries:

\starttyping
$ git log -p -2
commit ca82a6dff817ec66f44342007202690a93763949
Author: Scott Chacon <schacon@gee-mail.com>
Date:   Mon Mar 17 21:52:11 2008 -0700

    changed the version number

diff --git a/Rakefile b/Rakefile
index a874b73..8f94139 100644
--- a/Rakefile
+++ b/Rakefile
@@ -5,7 +5,7 @@ require 'rake/gempackagetask'
 spec = Gem::Specification.new do |s|
-    s.version   =   "0.1.0"
+    s.version   =   "0.1.1"
     s.author    =   "Scott Chacon"

commit 085bb3bcb608e1e8451d4b2432f8ecbe6306e7e7
Author: Scott Chacon <schacon@gee-mail.com>
Date:   Sat Mar 15 16:40:33 2008 -0700

    removed unnecessary test code

diff --git a/lib/simplegit.rb b/lib/simplegit.rb
index a0a60ae..47c6340 100644
--- a/lib/simplegit.rb
+++ b/lib/simplegit.rb
@@ -18,8 +18,3 @@ class SimpleGit
     end

 end
-
-if $0 == __FILE__
-  git = SimpleGit.new
-  puts git.show
-end
\ No newline at end of file
\stoptyping

This option displays the same information but with a diff directly
following each entry. This is very helpful for code review or to
quickly browse what happened during a series of commits that a
collaborator has added. You can also use a series of summarizing
options with \type{git log}. For example, if you want to see some
abbreviated stats for each commit, you can use the \type{--stat}
option:

\starttyping
$ git log --stat
commit ca82a6dff817ec66f44342007202690a93763949
Author: Scott Chacon <schacon@gee-mail.com>
Date:   Mon Mar 17 21:52:11 2008 -0700

    changed the version number

 Rakefile |    2 +-
 1 files changed, 1 insertions(+), 1 deletions(-)

commit 085bb3bcb608e1e8451d4b2432f8ecbe6306e7e7
Author: Scott Chacon <schacon@gee-mail.com>
Date:   Sat Mar 15 16:40:33 2008 -0700

    removed unnecessary test code

 lib/simplegit.rb |    5 -----
 1 files changed, 0 insertions(+), 5 deletions(-)

commit a11bef06a3f659402fe7563abf99ad00de2209e6
Author: Scott Chacon <schacon@gee-mail.com>
Date:   Sat Mar 15 10:31:28 2008 -0700

    first commit

 README           |    6 ++++++
 Rakefile         |   23 +++++++++++++++++++++++
 lib/simplegit.rb |   25 +++++++++++++++++++++++++
 3 files changed, 54 insertions(+), 0 deletions(-)
\stoptyping

As you can see, the \type{--stat} option prints below each commit
entry a list of modified files, how many files were changed, and
how many lines in those files were added and removed. It also puts
a summary of the information at the end. Another really useful
option is \type{--pretty}. This option changes the log output to
formats other than the default. A few prebuilt options are
available for you to use. The oneline option prints each commit on
a single line, which is useful if you’re looking at a lot of
commits. In addition, the \type{short}, \type{full}, and
\type{fuller} options show the output in roughly the same format
but with less or more information, respectively:

\starttyping
$ git log --pretty=oneline
ca82a6dff817ec66f44342007202690a93763949 changed the version number
085bb3bcb608e1e8451d4b2432f8ecbe6306e7e7 removed unnecessary test code
a11bef06a3f659402fe7563abf99ad00de2209e6 first commit
\stoptyping

The most interesting option is \type{format}, which allows you to
specify your own log output format. This is especially useful when
you’re generating output for machine parsing — because you specify
the format explicitly, you know it won’t change with updates to
Git:

\starttyping
$ git log --pretty=format:"%h - %an, %ar : %s"
ca82a6d - Scott Chacon, 11 months ago : changed the version number
085bb3b - Scott Chacon, 11 months ago : removed unnecessary test code
a11bef0 - Scott Chacon, 11 months ago : first commit
\stoptyping

Table 2--1 lists some of the more useful options that format takes.

\placetable[here]{Format options}
\starttable[|lp(0.09\textwidth)|lp(0.63\textwidth)|]
\HL
\NC Option
\NC Description of Output
\NC\AR
\HL
\NC \type{%H}
\NC Commit hash
\NC\AR
\NC \type{%h}
\NC Abbreviated commit hash
\NC\AR
\NC \type{%T}
\NC Tree hash
\NC\AR
\NC \type{%t}
\NC Abbreviated tree hash
\NC\AR
\NC \type{%P}
\NC Parent hashes
\NC\AR
\NC \type{%p}
\NC Abbreviated parent hashes
\NC\AR
\NC \type{%an}
\NC Author name
\NC\AR
\NC \type{%ae}
\NC Author e-mail
\NC\AR
\NC \type{%ad}
\NC Author date (format respects the \type{--date=} option)
\NC\AR
\NC \type{%ar}
\NC Author date, relative
\NC\AR
\NC \type{%cn}
\NC Committer name
\NC\AR
\NC \type{%ce}
\NC Committer email
\NC\AR
\NC \type{%cd}
\NC Committer date
\NC\AR
\NC \type{%cr}
\NC Committer date, relative
\NC\AR
\NC \type{%s}
\NC Subject
\NC\AR
\HL
\stoptable

You may be wondering what the difference is between {\em author}
and {\em committer}. The author is the person who originally wrote
the work, whereas the committer is the person who last applied the
work. So, if you send in a patch to a project and one of the core
members applies the patch, both of you get credit — you as the
author and the core member as the committer. We’ll cover this
distinction a bit more in Chapter 5.

The oneline and format options are particularly useful with another
\type{log} option called \type{--graph}. This option adds a nice
little ASCII graph showing your branch and merge history, which we
can see our copy of the Grit project repository:

\starttyping
$ git log --pretty=format:"%h %s" --graph
* 2d3acf9 ignore errors from SIGCHLD on trap
*  5e3ee11 Merge branch 'master' of git://github.com/dustin/grit
|\
| * 420eac9 Added a method for getting the current branch.
* | 30e367c timeout code and tests
* | 5a09431 add timeout protection to grit
* | e1193f8 support for heads with slashes in them
|/
* d6016bc require time for xmlschema
*  11d191e Merge branch 'defunkt' into local
\stoptyping

Those are only some simple output-formatting options to
\type{git log} — there are many more. Table 2--2 lists the options
we’ve covered so far and some other common formatting options that
may be useful, along with how they change the output of the log
command.

\placetable[here]{Some \type{git log} options}
\starttable[|lp(0.23\textwidth)|lp(0.75\textwidth)|]
\HL
\NC Option
\NC Description
\NC\AR
\HL
\NC \type{-p}
\NC Show the patch introduced with each commit.
\NC\AR
\NC \type{--stat}
\NC Show statistics for files modified in each commit.
\NC\AR
\NC \type{--shortstat}
\NC Display only the changed/insertions/deletions line from the
    \type{--stat} command.
\NC\AR
\NC \type{--name-only}
\NC Show the list of files modified after the commit information.
\NC\AR
\NC \type{--name-status}
\NC Show the list of files affected with added/modified/deleted
    information as well.
\NC\AR
\NC \type{--abbrev-commit}
\NC Show only the first few characters of the SHA--1 checksum
    instead of all 40.
\NC\AR
\NC \type{--relative-date}
\NC Display the date in a relative format (for example, “2 weeks
    ago”) instead of using the full date format.
\NC\AR
\NC \type{--graph}
\NC Display an ASCII graph of the branch and merge history beside
    the log output.
\NC\AR
\NC \type{--pretty}
\NC Show commits in an alternate format. Options include oneline,
    short, full, fuller, and format (where you specify your own
    format).
\NC\AR
\HL
\stoptable

\subsubsubject{Limiting Log Output}

In addition to output-formatting options, git log takes a number of
useful limiting options — that is, options that let you show only a
subset of commits. You’ve seen one such option already — the
\type{-2} option, which show only the last two commits. In fact,
you can do \type{-<n>}, where \type{n} is any integer to show the
last \type{n} commits. In reality, you’re unlikely to use that
often, because Git by default pipes all output through a pager so
you see only one page of log output at a time.

However, the time-limiting options such as \type{--since} and
\type{--until} are very useful. For example, this command gets the
list of commits made in the last two weeks:

\starttyping
$ git log --since=2.weeks
\stoptyping

This command works with lots of formats — you can specify a
specific date (“2008--01--15”) or a relative date such as “2 years
1 day 3 minutes ago”.

You can also filter the list to commits that match some search
criteria. The \type{--author} option allows you to filter on a
specific author, and the \type{--grep} option lets you search for
keywords in the commit messages. (Note that if you want to specify
both author and grep options, you have to add \type{--all-match} or
the command will match commits with either.)

The last really useful option to pass to \type{git log} as a filter
is a path. If you specify a directory or file name, you can limit
the log output to commits that introduced a change to those files.
This is always the last option and is generally preceded by double
dashes (\type{--}) to separate the paths from the options.

In Table 2--3 we’ll list these and a few other common options for
your reference.

\placetable[here]{Some \type{git log} filter options}
\starttable[|lp(0.28\textwidth)|lp(0.70\textwidth)|]
\HL
\NC Option
\NC Description
\NC\AR
\HL
\NC \type{-(n)}
\NC Show only the last n commits
\NC\AR
\NC \type{--since}, \type{--after}
\NC Limit the commits to those made after the specified date.
\NC\AR
\NC \type{--until}, \type{--before}
\NC Limit the commits to those made before the specified date.
\NC\AR
\NC \type{--author}
\NC Only show commits in which the author entry matches the
    specified string.
\NC\AR
\NC \type{--committer}
\NC Only show commits in which the committer entry matches the
    specified string.
\NC\AR
\HL
\stoptable

For example, if you want to see which commits modifying test files
in the Git source code history were committed by Junio Hamano and
were not merges in the month of October 2008, you can run something
like this:

\starttyping
$ git log --pretty="%h - %s" --author=gitster --since="2008-10-01" \
   --before="2008-11-01" --no-merges -- t/
5610e3b - Fix testcase failure when extended attribute
acd3b9e - Enhance hold_lock_file_for_{update,append}()
f563754 - demonstrate breakage of detached checkout wi
d1a43f2 - reset --hard/read-tree --reset -u: remove un
51a94af - Fix "checkout --track -b newbranch" on detac
b0ad11e - pull: allow "git pull origin $something:$cur
\stoptyping

Of the nearly 20,000 commits in the Git source code history, this
command shows the 6 that match those criteria.

\subsubsubject{Using a GUI to Visualize History}

If you like to use a more graphical tool to visualize your commit
history, you may want to take a look at a Tcl/Tk program called
gitk that is distributed with Git. Gitk is basically a visual
\type{git log} tool, and it accepts nearly all the filtering
options that \type{git log} does. If you type gitk on the command
line in your project, you should see something like Figure 2--2.

\placefigure[here,nonumber]{Figure 2--2. The gitk history visualizer.}{\externalfigure[../figures/18333fig0202-tn.png]}

You can see the commit history in the top half of the window along
with a nice ancestry graph. The diff viewer in the bottom half of
the window shows you the changes introduced at any commit you
click.

\subsubject{Undoing Things}

At any stage, you may want to undo something. Here, we’ll review a
few basic tools for undoing changes that you’ve made. Be careful,
because you can’t always undo some of these undos. This is one of
the few areas in Git where you may lose some work if you do it
wrong.

\subsubsubject{Changing Your Last Commit}

One of the common undos takes place when you commit too early and
possibly forget to add some files, or you mess up your commit
message. If you want to try that commit again, you can run commit
with the \type{--amend} option:

\starttyping
$ git commit --amend
\stoptyping

This command takes your staging area and uses it for the commit. If
you’ve have made no changes since your last commit (for instance,
you run this command immediately after your previous commit), then
your snapshot will look exactly the same and all you’ll change is
your commit message.

The same commit-message editor fires up, but it already contains
the message of your previous commit. You can edit the message the
same as always, but it overwrites your previous commit.

As an example, if you commit and then realize you forgot to stage
the changes in a file you wanted to add to this commit, you can do
something like this:

\starttyping
$ git commit -m 'initial commit'
$ git add forgotten_file
$ git commit --amend
\stoptyping

All three of these commands end up with a single commit — the
second commit replaces the results of the first.

\subsubsubject{Unstaging a Staged File}

The next two sections demonstrate how to wrangle your staging area
and working directory changes. The nice part is that the command
you use to determine the state of those two areas also reminds you
how to undo changes to them. For example, let’s say you’ve changed
two files and want to commit them as two separate changes, but you
accidentally type \type{git add *} and stage them both. How can you
unstage one of the two? The \type{git status} command reminds you:

\starttyping
$ git add .
$ git status
# On branch master
# Changes to be committed:
#   (use "git reset HEAD <file>..." to unstage)
#
#       modified:   README.txt
#       modified:   benchmarks.rb
#
\stoptyping

Right below the “Changes to be committed” text, it says use
\type{git reset HEAD <file>...} to unstage. So, let’s use that
advice to unstage the benchmarks.rb file:

\starttyping
$ git reset HEAD benchmarks.rb
benchmarks.rb: locally modified
$ git status
# On branch master
# Changes to be committed:
#   (use "git reset HEAD <file>..." to unstage)
#
#       modified:   README.txt
#
# Changed but not updated:
#   (use "git add <file>..." to update what will be committed)
#   (use "git checkout -- <file>..." to discard changes in working directory)
#
#       modified:   benchmarks.rb
#
\stoptyping

The command is a bit strange, but it works. The benchmarks.rb file
is modified but once again unstaged.

\subsubsubject{Unmodifying a Modified File}

What if you realize that you don’t want to keep your changes to the
benchmarks.rb file? How can you easily unmodify it — revert it back
to what it looked like when you last committed (or initially
cloned, or however you got it into your working directory)?
Luckily, \type{git status} tells you how to do that, too. In the
last example output, the unstaged area looks like this:

\starttyping
# Changed but not updated:
#   (use "git add <file>..." to update what will be committed)
#   (use "git checkout -- <file>..." to discard changes in working directory)
#
#       modified:   benchmarks.rb
#
\stoptyping

It tells you pretty explicitly how to discard the changes you’ve
made (at least, the newer versions of Git, 1.6.1 and later, do this
— if you have an older version, we highly recommend upgrading it to
get some of these nicer usability features). Let’s do what it says:

\starttyping
$ git checkout -- benchmarks.rb
$ git status
# On branch master
# Changes to be committed:
#   (use "git reset HEAD <file>..." to unstage)
#
#       modified:   README.txt
#
\stoptyping

You can see that the changes have been reverted. You should also
realize that this is a dangerous command: any changes you made to
that file are gone — you just copied another file over it. Don’t
ever use this command unless you absolutely know that you don’t
want the file. If you just need to get it out of the way, we’ll go
over stashing and branching in the next chapter; these are
generally better ways to go.

Remember, anything that is committed in Git can almost always be
recovered. Even commits that were on branches that were deleted or
commits that were overwritten with an \type{--amend} commit can be
recovered (see Chapter 9 for data recovery). However, anything you
lose that was never committed is likely never to be seen again.

\subsubject{Working with Remotes}

To be able to collaborate on any Git project, you need to know how
to manage your remote repositories. Remote repositories are
versions of your project that are hosted on the Internet or network
somewhere. You can have several of them, each of which generally is
either read-only or read/write for you. Collaborating with others
involves managing these remote repositories and pushing and pulling
data to and from them when you need to share work. Managing remote
repositories includes knowing how to add remote repositories,
remove remotes that are no longer valid, manage various remote
branches and define them as being tracked or not, and more. In this
section, we’ll cover these remote-management skills.

\subsubsubject{Showing Your Remotes}

To see which remote servers you have configured, you can run the
git remote command. It lists the shortnames of each remote handle
you’ve specified. If you’ve cloned your repository, you should at
least see origin — that is the default name Git gives to the server
you cloned from:

\starttyping
$ git clone git://github.com/schacon/ticgit.git
Initialized empty Git repository in /private/tmp/ticgit/.git/
remote: Counting objects: 595, done.
remote: Compressing objects: 100% (269/269), done.
remote: Total 595 (delta 255), reused 589 (delta 253)
Receiving objects: 100% (595/595), 73.31 KiB | 1 KiB/s, done.
Resolving deltas: 100% (255/255), done.
$ cd ticgit
$ git remote
origin
\stoptyping

You can also specify \type{-v}, which shows you the URL that Git
has stored for the shortname to be expanded to:

\starttyping
$ git remote -v
origin  git://github.com/schacon/ticgit.git
\stoptyping

If you have more than one remote, the command lists them all. For
example, my Grit repository looks something like this.

\starttyping
$ cd grit
$ git remote -v
bakkdoor  git://github.com/bakkdoor/grit.git
cho45     git://github.com/cho45/grit.git
defunkt   git://github.com/defunkt/grit.git
koke      git://github.com/koke/grit.git
origin    git@github.com:mojombo/grit.git
\stoptyping

This means we can pull contributions from any of these users pretty
easily. But notice that only the origin remote is an SSH URL, so
it’s the only one I can push to (we’ll cover why this is in Chapter
4).

\subsubsubject{Adding Remote Repositories}

I’ve mentioned and given some demonstrations of adding remote
repositories in previous sections, but here is how to do it
explicitly. To add a new remote Git repository as a shortname you
can reference easily, run \type{git remote add [shortname] [url]}:

\starttyping
$ git remote
origin
$ git remote add pb git://github.com/paulboone/ticgit.git
$ git remote -v
origin  git://github.com/schacon/ticgit.git
pb  git://github.com/paulboone/ticgit.git
\stoptyping

Now you can use the string pb on the command line in lieu of the
whole URL. For example, if you want to fetch all the information
that Paul has but that you don’t yet have in your repository, you
can run git fetch pb:

\starttyping
$ git fetch pb
remote: Counting objects: 58, done.
remote: Compressing objects: 100% (41/41), done.
remote: Total 44 (delta 24), reused 1 (delta 0)
Unpacking objects: 100% (44/44), done.
From git://github.com/paulboone/ticgit
 * [new branch]      master     -> pb/master
 * [new branch]      ticgit     -> pb/ticgit
\stoptyping

Paul’s master branch is accessible locally as \type{pb/master} —
you can merge it into one of your branches, or you can check out a
local branch at that point if you want to inspect it.

\subsubsubject{Fetching and Pulling from Your Remotes}

As you just saw, to get data from your remote projects, you can
run:

\starttyping
$ git fetch [remote-name]
\stoptyping

The command goes out to that remote project and pulls down all the
data from that remote project that you don’t have yet. After you do
this, you should have references to all the branches from that
remote, which you can merge in or inspect at any time. (We’ll go
over what branches are and how to use them in much more detail in
Chapter 3.)

If you clone a repository, the command automatically adds that
remote repository under the name origin. So,
\type{git fetch origin} fetches any new work that has been pushed
to that server since you cloned (or last fetched from) it. It’s
important to note that the fetch command pulls the data to your
local repository — it doesn’t automatically merge it with any of
your work or modify what you’re currently working on. You have to
merge it manually into your work when you’re ready.

If you have a branch set up to track a remote branch (see the next
section and Chapter 3 for more information), you can use the
\type{git pull} command to automatically fetch and then merge a
remote branch into your current branch. This may be an easier or
more comfortable workflow for you; and by default, the
\type{git clone} command automatically sets up your local master
branch to track the remote master branch on the server you cloned
from (assuming the remote has a master branch). Running
\type{git pull} generally fetches data from the server you
originally cloned from and automatically tries to merge it into the
code you’re currently working on.

\subsubsubject{Pushing to Your Remotes}

When you have your project at a point that you want to share, you
have to push it upstream. The command for this is simple:
\type{git push [remote-name] [branch-name]}. If you want to push
your master branch to your \type{origin} server (again, cloning
generally sets up both of those names for you automatically), then
you can run this to push your work back up to the server:

\starttyping
$ git push origin master
\stoptyping

This command works only if you cloned from a server to which you
have write access and if nobody has pushed in the meantime. If you
and someone else clone at the same time and they push upstream and
then you push upstream, your push will rightly be rejected. You’ll
have to pull down their work first and incorporate it into yours
before you’ll be allowed to push. See Chapter 3 for more detailed
information on how to push to remote servers.

\subsubsubject{Inspecting a Remote}

If you want to see more information about a particular remote, you
can use the \type{git remote show [remote-name]} command. If you
run this command with a particular shortname, such as
\type{origin}, you get something like this:

\starttyping
$ git remote show origin
* remote origin
  URL: git://github.com/schacon/ticgit.git
  Remote branch merged with 'git pull' while on branch master
    master
  Tracked remote branches
    master
    ticgit
\stoptyping

It lists the URL for the remote repository as well as the tracking
branch information. The command helpfully tells you that if you’re
on the master branch and you run \type{git pull}, it will
automatically merge in the master branch on the remote after it
fetches all the remote references. It also lists all the remote
references it has pulled down.

That is a simple example you’re likely to encounter. When you’re
using Git more heavily, however, you may see much more information
from \type{git remote show}:

\starttyping
$ git remote show origin
* remote origin
  URL: git@github.com:defunkt/github.git
  Remote branch merged with 'git pull' while on branch issues
    issues
  Remote branch merged with 'git pull' while on branch master
    master
  New remote branches (next fetch will store in remotes/origin)
    caching
  Stale tracking branches (use 'git remote prune')
    libwalker
    walker2
  Tracked remote branches
    acl
    apiv2
    dashboard2
    issues
    master
    postgres
  Local branch pushed with 'git push'
    master:master
\stoptyping

This command shows which branch is automatically pushed when you
run \type{git push} on certain branches. It also shows you which
remote branches on the server you don’t yet have, which remote
branches you have that have been removed from the server, and
multiple branches that are automatically merged when you run
\type{git pull}.

\subsubsubject{Removing and Renaming Remotes}

If you want to rename a reference, in newer versions of Git you can
run \type{git remote rename} to change a remote’s shortname. For
instance, if you want to rename \type{pb} to \type{paul}, you can
do so with \type{git remote rename}:

\starttyping
$ git remote rename pb paul
$ git remote
origin
paul
\stoptyping

It’s worth mentioning that this changes your remote branch names,
too. What used to be referenced at \type{pb/master} is now at
\type{paul/master}.

If you want to remove a reference for some reason — you’ve moved
the server or are no longer using a particular mirror, or perhaps a
contributor isn’t contributing anymore — you can use
\type{git remote rm}:

\starttyping
$ git remote rm paul
$ git remote
origin
\stoptyping

\subsubject{Tagging}

Like most VCSs, Git has the ability to tag specific points in
history as being important. Generally, people use this
functionality to mark release points (v1.0, and so on). In this
section, you’ll learn how to list the available tags, how to create
new tags, and what the different types of tags are.

\subsubsubject{Listing Your Tags}

Listing the available tags in Git is straightforward. Just type
\type{git tag}:

\starttyping
$ git tag
v0.1
v1.3
\stoptyping

This command lists the tags in alphabetical order; the order in
which they appear has no real importance.

You can also search for tags with a particular pattern. The Git
source repo, for instance, contains more than 240 tags. If you’re
only interested in looking at the 1.4.2 series, you can run this:

\starttyping
$ git tag -l 'v1.4.2.*'
v1.4.2.1
v1.4.2.2
v1.4.2.3
v1.4.2.4
\stoptyping

\subsubsubject{Creating Tags}

Git uses two main types of tags: lightweight and annotated. A
lightweight tag is very much like a branch that doesn’t change —
it’s just a pointer to a specific commit. Annotated tags, however,
are stored as full objects in the Git database. They’re
checksummed; contain the tagger name, e-mail, and date; have a
tagging message; and can be signed and verified with GNU Privacy
Guard (GPG). It’s generally recommended that you create annotated
tags so you can have all this information; but if you want a
temporary tag or for some reason don’t want to keep the other
information, lightweight tags are available too.

\subsubsubject{Annotated Tags}

Creating an annotated tag in Git is simple. The easiest way is to
specify \type{-a} when you run the \type{tag} command:

\starttyping
$ git tag -a v1.4 -m 'my version 1.4'
$ git tag
v0.1
v1.3
v1.4
\stoptyping

The \type{-m} specifies a tagging message, which is stored with the
tag. If you don’t specify a message for an annotated tag, Git
launches your editor so you can type it in.

You can see the tag data along with the commit that was tagged by
using the \type{git show} command:

\starttyping
$ git show v1.4
tag v1.4
Tagger: Scott Chacon <schacon@gee-mail.com>
Date:   Mon Feb 9 14:45:11 2009 -0800

my version 1.4
commit 15027957951b64cf874c3557a0f3547bd83b3ff6
Merge: 4a447f7... a6b4c97...
Author: Scott Chacon <schacon@gee-mail.com>
Date:   Sun Feb 8 19:02:46 2009 -0800

    Merge branch 'experiment'
\stoptyping

That shows the tagger information, the date the commit was tagged,
and the annotation message before showing the commit information.

\subsubsubject{Signed Tags}

You can also sign your tags with GPG, assuming you have a private
key. All you have to do is use \type{-s} instead of \type{-a}:

\starttyping
$ git tag -s v1.5 -m 'my signed 1.5 tag'
You need a passphrase to unlock the secret key for
user: "Scott Chacon <schacon@gee-mail.com>"
1024-bit DSA key, ID F721C45A, created 2009-02-09
\stoptyping

If you run \type{git show} on that tag, you can see your GPG
signature attached to it:

\starttyping
$ git show v1.5
tag v1.5
Tagger: Scott Chacon <schacon@gee-mail.com>
Date:   Mon Feb 9 15:22:20 2009 -0800

my signed 1.5 tag
-----BEGIN PGP SIGNATURE-----
Version: GnuPG v1.4.8 (Darwin)

iEYEABECAAYFAkmQurIACgkQON3DxfchxFr5cACeIMN+ZxLKggJQf0QYiQBwgySN
Ki0An2JeAVUCAiJ7Ox6ZEtK+NvZAj82/
=WryJ
-----END PGP SIGNATURE-----
commit 15027957951b64cf874c3557a0f3547bd83b3ff6
Merge: 4a447f7... a6b4c97...
Author: Scott Chacon <schacon@gee-mail.com>
Date:   Sun Feb 8 19:02:46 2009 -0800

    Merge branch 'experiment'
\stoptyping

A bit later, you’ll learn how to verify signed tags.

\subsubsubject{Lightweight Tags}

Another way to tag commits is with a lightweight tag. This is
basically the commit checksum stored in a file — no other
information is kept. To create a lightweight tag, don’t supply the
\type{-a}, \type{-s}, or \type{-m} option:

\starttyping
$ git tag v1.4-lw
$ git tag
v0.1
v1.3
v1.4
v1.4-lw
v1.5
\stoptyping

This time, if you run \type{git show} on the tag, you don’t see the
extra tag information. The command just shows the commit:

\starttyping
$ git show v1.4-lw
commit 15027957951b64cf874c3557a0f3547bd83b3ff6
Merge: 4a447f7... a6b4c97...
Author: Scott Chacon <schacon@gee-mail.com>
Date:   Sun Feb 8 19:02:46 2009 -0800

    Merge branch 'experiment'
\stoptyping

\subsubsubject{Verifying Tags}

To verify a signed tag, you use \type{git tag -v [tag-name]}. This
command uses GPG to verify the signature. You need the signer’s
public key in your keyring for this to work properly:

\starttyping
$ git tag -v v1.4.2.1
object 883653babd8ee7ea23e6a5c392bb739348b1eb61
type commit
tag v1.4.2.1
tagger Junio C Hamano <junkio@cox.net> 1158138501 -0700

GIT 1.4.2.1

Minor fixes since 1.4.2, including git-mv and git-http with alternates.
gpg: Signature made Wed Sep 13 02:08:25 2006 PDT using DSA key ID F3119B9A
gpg: Good signature from "Junio C Hamano <junkio@cox.net>"
gpg:                 aka "[jpeg image of size 1513]"
Primary key fingerprint: 3565 2A26 2040 E066 C9A7  4A7D C0C6 D9A4 F311 9B9A
\stoptyping

If you don’t have the signer’s public key, you get something like
this instead:

\starttyping
gpg: Signature made Wed Sep 13 02:08:25 2006 PDT using DSA key ID F3119B9A
gpg: Can't check signature: public key not found
error: could not verify the tag 'v1.4.2.1'
\stoptyping

\subsubsubject{Tagging Later}

You can also tag commits after you’ve moved past them. Suppose your
commit history looks like this:

\starttyping
$ git log --pretty=oneline
15027957951b64cf874c3557a0f3547bd83b3ff6 Merge branch 'experiment'
a6b4c97498bd301d84096da251c98a07c7723e65 beginning write support
0d52aaab4479697da7686c15f77a3d64d9165190 one more thing
6d52a271eda8725415634dd79daabbc4d9b6008e Merge branch 'experiment'
0b7434d86859cc7b8c3d5e1dddfed66ff742fcbc added a commit function
4682c3261057305bdd616e23b64b0857d832627b added a todo file
166ae0c4d3f420721acbb115cc33848dfcc2121a started write support
9fceb02d0ae598e95dc970b74767f19372d61af8 updated rakefile
964f16d36dfccde844893cac5b347e7b3d44abbc commit the todo
8a5cbc430f1a9c3d00faaeffd07798508422908a updated readme
\stoptyping

Now, suppose you forgot to tag the project at v1.2, which was at
the \quotation{updated rakefile} commit. You can add it after the
fact. To tag that commit, you specify the commit checksum (or part
of it) at the end of the command:

\starttyping
$ git tag -a v1.2 9fceb02
\stoptyping

You can see that you’ve tagged the commit:

\starttyping
$ git tag
v0.1
v1.2
v1.3
v1.4
v1.4-lw
v1.5

$ git show v1.2
tag v1.2
Tagger: Scott Chacon <schacon@gee-mail.com>
Date:   Mon Feb 9 15:32:16 2009 -0800

version 1.2
commit 9fceb02d0ae598e95dc970b74767f19372d61af8
Author: Magnus Chacon <mchacon@gee-mail.com>
Date:   Sun Apr 27 20:43:35 2008 -0700

    updated rakefile
...
\stoptyping

\subsubsubject{Sharing Tags}

By default, the \type{git push} command doesn’t transfer tags to
remote servers. You will have to explicitly push tags to a shared
server after you have created them. This process is just like
sharing remote branches — you can run
\type{git push origin [tagname]}.

\starttyping
$ git push origin v1.5
Counting objects: 50, done.
Compressing objects: 100% (38/38), done.
Writing objects: 100% (44/44), 4.56 KiB, done.
Total 44 (delta 18), reused 8 (delta 1)
To git@github.com:schacon/simplegit.git
* [new tag]         v1.5 -> v1.5
\stoptyping

If you have a lot of tags that you want to push up at once, you can
also use the \type{--tags} option to the \type{git push} command.
This will transfer all of your tags to the remote server that are
not already there.

\starttyping
$ git push origin --tags
Counting objects: 50, done.
Compressing objects: 100% (38/38), done.
Writing objects: 100% (44/44), 4.56 KiB, done.
Total 44 (delta 18), reused 8 (delta 1)
To git@github.com:schacon/simplegit.git
 * [new tag]         v0.1 -> v0.1
 * [new tag]         v1.2 -> v1.2
 * [new tag]         v1.4 -> v1.4
 * [new tag]         v1.4-lw -> v1.4-lw
 * [new tag]         v1.5 -> v1.5
\stoptyping

Now, when someone else clones or pulls from your repository, they
will get all your tags as well.

\subsubject{Tips and Tricks}

Before we finish this chapter on basic Git, a few little tips and
tricks may make your Git experience a bit simpler, easier, or more
familiar. Many people use Git without using any of these tips, and
we won’t refer to them or assume you’ve used them later in the
book; but you should probably know how to do them.

\subsubsubject{Auto-Completion}

If you use the Bash shell, Git comes with a nice auto-completion
script you can enable. Download the Git source code, and look in
the \type{contrib/completion} directory; there should be a file
called \type{git-completion.bash}. Copy this file to your home
directory, and add this to your \type{.bashrc} file:

\starttyping
source ~/.git-completion.bash
\stoptyping

If you want to set up Git to automatically have Bash shell
completion for all users, copy this script to the
\type{/opt/local/etc/bash_completion.d} directory on Mac systems or
to the \type{/etc/bash_completion.d/} directory on Linux systems.
This is a directory of scripts that Bash will automatically load to
provide shell completions.

If you’re using Windows with Git Bash, which is the default when
installing Git on Windows with msysGit, auto-completion should be
preconfigured.

Press the Tab key when you’re writing a Git command, and it should
return a set of suggestions for you to pick from:

\starttyping
$ git co<tab><tab>
commit config
\stoptyping

In this case, typing git co and then pressing the Tab key twice
suggests commit and config. Adding \type{m<tab>} completes
\type{git commit} automatically.

This also works with options, which is probably more useful. For
instance, if you’re running a \type{git log} command and can’t
remember one of the options, you can start typing it and press Tab
to see what matches:

\starttyping
$ git log --s<tab>
--shortstat  --since=  --src-prefix=  --stat   --summary
\stoptyping

That’s a pretty nice trick and may save you some time and
documentation reading.

\subsubsubject{Git Aliases}

Git doesn’t infer your command if you type it in partially. If you
don’t want to type the entire text of each of the Git commands, you
can easily set up an alias for each command using
\type{git config}. Here are a couple of examples you may want to
set up:

\starttyping
$ git config --global alias.co checkout
$ git config --global alias.br branch
$ git config --global alias.ci commit
$ git config --global alias.st status
\stoptyping

This means that, for example, instead of typing \type{git commit},
you just need to type \type{git ci}. As you go on using Git, you’ll
probably use other commands frequently as well; in this case, don’t
hesitate to create new aliases.

This technique can also be very useful in creating commands that
you think should exist. For example, to correct the usability
problem you encountered with unstaging a file, you can add your own
unstage alias to Git:

\starttyping
$ git config --global alias.unstage 'reset HEAD --'
\stoptyping

This makes the following two commands equivalent:

\starttyping
$ git unstage fileA
$ git reset HEAD fileA
\stoptyping

This seems a bit clearer. It’s also common to add a \type{last}
command, like this:

\starttyping
$ git config --global alias.last 'log -1 HEAD'
\stoptyping

This way, you can see the last commit easily:

\starttyping
$ git last
commit 66938dae3329c7aebe598c2246a8e6af90d04646
Author: Josh Goebel <dreamer3@example.com>
Date:   Tue Aug 26 19:48:51 2008 +0800

    test for current head

    Signed-off-by: Scott Chacon <schacon@example.com>
\stoptyping

As you can tell, Git simply replaces the new command with whatever
you alias it for. However, maybe you want to run an external
command, rather than a Git subcommand. In that case, you start the
command with a \type{!} character. This is useful if you write your
own tools that work with a Git repository. We can demonstrate by
aliasing \type{git visual} to run \type{gitk}:

\starttyping
$ git config --global alias.visual "!gitk"
\stoptyping

\subsubject{Summary}

At this point, you can do all the basic local Git operations —
creating or cloning a repository, making changes, staging and
committing those changes, and viewing the history of all the
changes the repository has been through. Next, we’ll cover Git’s
killer feature: its branching model.

\stoptext
